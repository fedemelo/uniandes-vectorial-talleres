\documentclass{fmbvecto}

\usepackage[spanish]{babel}

\renewcommand{\title}{Taller 3}
\newcommand{\subject}{Cálculo Vectorial}

\renewcommand{\labelenumii}{\theenumii}
\renewcommand{\theenumii}{\theenumi.\arabic{enumii}.}

\NewDocumentCommand{\itemp}{o}{\item (#1 puntos)}

\begin{document}

% TODO: Fecha

\begin{center}
    \textbf{\LARGE \title} \\
    {\large \subject}
\end{center}


Profesor: Jacinto Eloy Puig Portal, \href{mailto:jpuig@uniandes.edu.co}{jpuig@uniandes.edu.co}. \\
Monitor: Federico Melo Barrero, \href{mailto:f.melo@uniandes.edu.co}{f.melo@uniandes.edu.co}.\\

\textbf{\Large Preámbulo}

Las instrucciones referentes a la entrega del taller están escritas en \href{https://bloqueneon.uniandes.edu.co/d2l/home}{Bloque Neón}.

\subsection*{Bonos}
\begin{itemize}
  \item Se sumarán 0.25 puntos de bonificación a la nota del taller si su contenido está ordenado y puede leerse con facilidad.
  \item Se sumarán 0.25 puntos de bonificación a la nota del taller si no contiene errores léxicos, gramaticales ni faltas de ortografía.
\end{itemize}
La nota del taller puede exceder el 5.0.

\subsection*{Recomendaciones}

No necesita hacer uso de herramientas que le ayuden a hacer matemáticas, ya sean calculadoras, aplicaciones, grandes modelos de lenguaje u otras. Le recomiendo que no lo haga

Recuerde incluir las unidades siempre que trate con magnitudes físicas.

\section{Taller 2}

(1 punto) Corrija todos los errores que tuvo en el taller 2. Si no tuvo errores, omita este punto.


\end{document}
