\documentclass{fmbvecto}

\usepackage[spanish]{babel}

\renewcommand{\title}{Taller 2}
\newcommand{\subject}{Cálculo Vectorial}

\renewcommand{\labelenumii}{\theenumii}
\renewcommand{\theenumii}{\theenumi.\arabic{enumii}.}

\NewDocumentCommand{\itemp}{o}{\item (#1 puntos)}

\begin{document}

TODO: Fecha. Jueves, 13 de junio de 2024

\begin{center}
    \textbf{\LARGE \title} \\
    {\large \subject}
\end{center}


Profesor: Jacinto Eloy Puig Portal, \href{mailto:jpuig@uniandes.edu.co}{jpuig@uniandes.edu.co}. \\
Monitor: Federico Melo Barrero, \href{mailto:f.melo@uniandes.edu.co}{f.melo@uniandes.edu.co}.\\

\textbf{\Large Preámbulo}

Las instrucciones referentes a la entrega del taller están escritas en \href{https://bloqueneon.uniandes.edu.co/d2l/home}{Bloque Neón}.

\textbf{Bonos}
\begin{itemize}
  \item Se sumarán 0.5 puntos de bonificación a la nota del taller si su contenido está ordenado y puede leerse con facilidad.
  \item Se sumarán 0.5 puntos de bonificación a la nota del taller si no contiene errores léxicos, gramaticales ni faltas de ortografía.
\end{itemize}
La nota del taller puede exceder el 5.0.

\textbf{Recomendaciones}

No necesita hacer uso de herramientas que le ayuden a hacer matemáticas, ya sean calculadoras, aplicaciones, grandes modelos de lenguaje u otras. Le recomiendo que no lo haga

Recuerde incluir las unidades siempre que trate con magnitudes físicas.

\section{Taller 1}

\begin{enumerate}
    \item (1 punto) Corrija todos los errores que tuvo su grupo en el taller 1. Si no tuvo errores, omita este punto.
\end{enumerate}

\section{Integrales dobles}

    \begin{enumerate}
        \setcounter{enumi}{1}
        \item (1 punto) Calcule analíticamente la integral \[ \int_{1}^{2} \int_{0}^{\log y} (y-1) \sqrt{1 + \mathrm{e}^{2x}} \: \mathrm{d}x \: \mathrm{d}y. \] No utilice aproximaciones numéricas ni herramientas computacionales. No omita ningún paso en su solución.
    \end{enumerate}


\end{document}
